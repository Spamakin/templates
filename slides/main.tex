% Mirror: https://github.com/SIGma-UIUC/presentation-format
% --------------------------------------------------------------------
% This is a simple Beamer document that uses beamerthemesigma.sty
% Reading the comments should help you create a presentation even if
% you've never used Beamer before.
% --------------------------------------------------------------------

% Set our document class to Beamer
\documentclass[aspectratio=169]{beamer}
% \documentclass[aspectratio=169, handout]{beamer}
% Add handout option to ignore pauses

% From Jeff E
\usepackage{algo}
% Some more macros
\usepackage{sigmastyle}


% Set a title
\title{A Sample \TeX\ SIGma Presentation}

% Set a subtitle if you desire
\subtitle{[TAOCP 5 8.9.10.11]}

% Whoever worked on the presentation:
\author{Ma, Sig}

% Date looks ugly, so leave blank
\date{}

% An institute name, if you're so inclined
% \institute{University of Illinois Urbana-Champaign}

% Use the SIGma theme for this Beamer presentation
\usetheme{sigma}
% --------------------------------------------------------------------

% Begin document
\begin{document}

% Beamer calls each slide a "frame", defined within the environment:
% \begin{frame}
%   <frame content here>
% \end{frame}

% This frame is just the title.
\begin{frame}
\titlepage
\end{frame}

% A frame with the table of contents.
% This frame's title is "Outline".
\begin{frame}{Outline}
  \tableofcontents
\end{frame}

\begin{frame}{Updates!}
  % Let's put some real content in this frame:
  Weekly updates:
  \begin{itemize}
    \item SIGma is an excellent SIG.
    \item I'm out of ideas for updates.
  \end{itemize}
\end{frame}

% Start a section: *sections* (subsections, etc.) are what show up in the TOC.
\section{Basics}
% Section pages can be printed thus:
\frame{\sectionpage}
% There's a way to automate this, see:
% https://tex.stackexchange.com/questions/178800/creating-sections-each-with-title-pages-in-beamers-slides/178803

\begin{frame}{Some Text}
    \begin{itemize}
        \item You may want some stuff to appear in a sequence \pause
        \item Use \textbackslash pause for this \pause
        \item \textcolor{sigma@mainblue}{colors} \textcolor{sigma@highlightpink}{are} \textcolor{sigma@alertred}{cool}
    \end{itemize}
\end{frame}

\begin{frame}
  \frametitle{Some Math Mode Testing}
  % Some fun with LaTeX Math
  $$\frac{x^2+3}{y^2+7}$$

  \[
    \mathcal L_{\mathcal T}(\vec{\lambda})
    = \sum_{(\mathbf{x},\mathbf{s})\in \mathcal T}
       \log P(\mathbf{s}\mid\mathbf{x}) - \sum_{i=1}^m
       \frac{\lambda_i^2}{2\sigma^2}
  \]

  $$\int_0^8 f(x) dx$$
\end{frame}

% Use \pause to make stuff readable
% Large walls of text scare the audience, we don't want that
% Introducing stuff sequentially allows for questions
\begin{frame}
  % Alternate syntax for frame titles
  \frametitle{There Is No Largest Prime Number}
  % Frames can have subtitles:
  \framesubtitle{The proof uses \textit{reductio ad absurdum}.}
  % Some frame content:
  \begin{thrm}
    There is no largest prime number.
  \end{thrm}
  \begin{pf}
    \begin{enumerate}
        \item Suppose $p$ were the largest prime number \pause
        \item Let $q$ be the product of the first $p$ primes \pause
        \item Then $q+1$ is not divisible by any of them \pause
        \item But $q + 1$ is greater than $1$, thus divisible by some prime number not in the first $p$ numbers. \pause
        \item Thus, there exists a prime larger than $p$.
    \end{enumerate}
  \end{pf}
\end{frame}

% However, this doesn't work in math mode. It is quite annoying to figure out
% So just copy this as reference
% This works for \onslide<> and \item<>
% Really good read on this: 
%   https://www.texdev.net/2014/01/17/the-beamer-slide-overlay-concept/
\begin{frame}{Sequential Math Frames}
    Here is a sentence \pause
    
    I shall now carry out some calculations \pause
    \begin{align*}
        \onslide<+->{\zeta(s) &= \sum_{n = 1}^\infty \frac{1}{n^s} \\}
        \onslide<+->{&= \prod_{p \in \text{primes}} \frac{1}{1 - p^{-s}} \\}
        \onslide<.->{&= \frac{1}{1 - 2^-s} \cdot \frac{1}{1 - 3^-s} \cdots \\}
        \onslide<+->{&= \frac{1}{\Gamma(s)} \int_0^\infty \frac{x^{s - 1}}{e^x - 1} ~\textrm{d}x\\}
    \end{align*}
\end{frame}

\section{Some Template Slides}
\frame{\sectionpage}

% Similar for subsections:
\subsection{A subsection, Wow}
% And their pages:
\frame{\subsectionpage}

\begin{frame}{Image}
  % This is how you'd include an image, centered.
  \begin{center}
    \includegraphics[width=0.25\textwidth]{images/sigma.png}
  \end{center}
\end{frame}

\begin{frame}{Side by Side}
    \includegraphics[width=0.25\textwidth]{images/sigma.png}\hspace{0.4\textwidth}
    \includegraphics[width=0.25\textwidth]{images/sigma.png}
\end{frame}

\begin{frame}{Demonstration of algo and nalgo env}
    \begin{algo}
    \underline{\textsc{GetRandomNumber}():}\+
    \\      return $4$   \Comment{chosen by fair dice roll.}
    \\      \hspace{42.75pt}\Comment{guaranteed to be random.}
    \end{algo}
    
    % nalgo has line numbers
    % only lines with \label{} are numbered
    \begin{nalgo}[1.3]
        \underline{\textsc{GetRandomNumber}():}\+
    \\\label{}  return $4$   \Comment{chosen by fair dice roll.}
    \\\label{}  \hspace{42.75pt}\Comment{guaranteed to be random.}
    \end{nalgo}
    
    Random number generation from~\cite{site:xkcd}. 
    \quest{
    \textbackslash cref for line numbers does not work. 
    If you want to refer to specific line numbers, do it manually
    }
    
\end{frame}

\begin{frame}{}
\begin{minipage}[c]{0.6\textwidth}
\begin{nalgo}
\textul{\textbf{\textsc{AlgorithmP}}$(S[a_0, \dots, a_n])$}:
\\\label{}  $C[1..n] \gets 0, O[1..n] \gets 1$
\\\label{}  \textbf{while} \textsc{True}:\+
\\\label{}      \textsc{print}$(S)$
\\\label{}      {\color{lightgray}$j \gets n, s \gets 0$}
\\\label{}      {\color{lightgray}\textbf{A:}~~$q \gets C[j] + O[j]$\+}
\\\label{}          {\color{lightgray}\textbf{if} $q < 0$: goto \textbf{D}}
\\\label{}          {\color{lightgray}\textbf{if} $q = j$: goto \textbf{B}}
\\\label{}          {\color{lightgray}\textsc{swap}$(S, j-C[j]+s, j-q+s)$}
\\\label{}          {\color{lightgray}$C[j] \gets q$} 
\\\label{}          {\color{lightgray}continue\-}
\\\label{}      {\color{lightgray}\textbf{B:}~~\textbf{if} $j=1$:\+\+}
\\\label{}              {\color{lightgray}break\-}
\\\label{}          {\color{lightgray}$s \gets s+1$\-}
\\\label{}    {\color{lightgray}\textbf{D:}~~$O[j] \gets -O[j], j \gets j-1$\+}
\\\label{}      {\color{lightgray}goto \textbf{A}}
\end{nalgo}
\end{minipage}
\begin{minipage}[c]{0.35\textwidth}
\begin{itemize}
    \item Here is an example of annotating an algorithms \pause
    \item We have grayed out text to highlight what we want to discuss 
\end{itemize}
\end{minipage}
\end{frame}

% Of course, not everything is in pseudocode
% You MUST have this [containsverbatim] option
% https://www.overleaf.com/learn/latex/Code_Highlighting_with_minted
\begin{frame}[containsverbatim]{Source Code}
    \begin{minted}
    [
    framesep=2mm,
    baselinestretch=1.2,
    bgcolor=black,
    fontsize=\footnotesize,
    ]
    {python}
    def algorithm_g(n):
        a = [0 for _ in range(n + 1)]
        while True:
            curr = a[1 : n + 1][::-1]
            yield "".join([str(i) for i in curr])
    
            a[0] = 1 - a[0]
            j = 1
            while a[j - 1] != 1:
                j += 1
            if j == n + 1:
                return
            a[j] = 1 - a[j]
    \end{minted}
\end{frame}


\begin{frame}{Theorems and Lemmas}
    \begin{lem}
        The map $\C \times \pqty{\C \setminus \set{0}} \to \C$, $(z, w) \mapsto z / w$ is $C^{\infty}$.
    \end{lem}
    \begin{pf}
        To see this, we identify $\C$ with $\R^{2}$ where $a + bi = (a, b)$.
        Thus our map is now
        \begin{align*}
            \pqty{\R^{2}} \times \pqty{\R^{2} \setminus \set{(0, 0)}} &\to \R^{2} \\
                            ((a, b), (c, d)) &\mapsto \pqty{\frac{ac + bd}{c^{2} + d^{2}}, \frac{bc - ad}{c^{2} + d^{2}}}
        \end{align*}
        which is well defined since at least one of $c$ or $d$ is nonzero.
        It is simple to verify that this map is equivalent to the original one.
        Since this new map is $C^{\infty}$ in each component, we have that it is $C^{\infty}$ overall.
    \end{pf}
\end{frame}

\section{Conclusion}
\frame{\sectionpage}

% Asking questions is fun but we should answer some first
\begin{frame}{}
      \begin{center}
    {\color{sigma@mainblue} \LARGE Questions?}
  \end{center}
\end{frame}

\begin{frame}{Questions!}
    \begin{itemize}
        \item How many zeros of $\zeta(s) = \sum\limits_{i = 1}^\infty \frac{1}{n^s} = \prod\limits_{p \text{ prime}} \frac{1}{1 - p^{-s}}$ have real part equal to $\frac{1}{2}$?
        \item Find a closed form to the following recurrence:
        $$
            f(n) = \begin{cases} 
                f\pqty{\frac{n}{2}} & n \text{ even} \\
                f{3n + 1} & \text{otherwise}                
            \end{cases}
        $$
    \end{itemize}
\end{frame}

% Quotes are fun, find some to use!
\font\eightss=cmssq8
\font\eightssi=cmssqi8
\newcommand\quoteAuthorDate[3]{\begingroup
  \baselineskip 10pt
  \parfillskip 0pt
  \interlinepenalty 10000 % not needed in example
  \leftskip 0pt plus 40pc minus \parindent
  \let\rm=\eightss
  \let\sl=\eightssi
  \everypar{\sl}#1\par
  \nobreak\smallskip
  \noindent\rm--- #2\unskip\enspace(#3)\par
  \endgroup}
% If someone can figure out how to horizontally center this and make the text bigger that'd be cool
\begin{frame}
    \begin{center}
        \item \quoteAuthorDate{So long and thanks for all the fish!}{DOUGLAS ADAMS}{\color{sigma@mainblue} 1979}
    \end{center}
\end{frame}

% Remove this slide if you came up with all the material yourself
\begin{frame}{Bibliography}
    \bibliography{refs}
    \bibliographystyle{alpha}
\end{frame}

\end{document}
